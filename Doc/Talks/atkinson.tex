\documentclass[10pt]{article}
\usepackage{amstex,amssymb}
\usepackage{theorem}

\newtheorem{theorem}{Theorem}
\newtheorem{proposition}{Proposition}
\newtheorem{definition}{Definition}
\newtheorem{corollary}{Corollary}
\newtheorem{lemma}{Lemma}
\newtheorem{conjecture}{Conjecture}
{\theorembodyfont{\rmfamily} \newtheorem{ex}{Example}}
\newenvironment{proof}{\noindent\textsc{Proof}}{\hfill\ensuremath{\blacksquare}}
\newenvironment{remark}{\noindent\textbf{Remark}}{}
\newcommand{\rad}{\textup{rad}}
\parskip 0.1in
\parindent 0pt

\begin{document}

\date{}

\title{The  $p$-Modular Descent Algebras}
\author{M.D. Atkinson, S.J. van Willigenburg, G. Pfeiffer }
\maketitle

Descent algebras are non-commutative, non-semi-simple algebras
associated with Coxeter groups.  They were first discovered by Solomon
in the 1970's and for the last 10 years have been studied
intensively.  Previous work has concentrated on the case that the
underlying field has characteristic zero.  However, characteristic $p$
analogues exist and their structure is very
sensitive to the value of $p$.  Our work determines the radical of
a descent algebra in characteristic $p$, and the irreducible
modules.  It also explains how the representation theory is connected
to the representation theory in characteristic zero.

Almost none of this work requires computation.  On the other hand almost
none of it could have been done without computation.  We wrote a number of
ad hoc programs to enable us to work within particular descent algebras,
these programs suggested various theorems, and we then managed to prove
the theorems ``by hand''.


Let $W$ be a Coxeter group with generating set $S$ of fundamental
reflections. Thus every element $w\in W$ can be written as a product
of elements in $S$; we let $\lambda(w)$ denote the length of a
shortest expression for $w$.  If $L$ is any subset of $S$ let
$W_L$ be the subgroup generated by $L$; $W_{L}$ is normally called a
parabolic subgroup of $W$.  Let $X_L$ be the (unique) set of
minimal length representatives of the left cosets
of $W_L$ in $W$.

Solomon proved the following remarkable theorem:
\begin{theorem}\cite{solomon-mackey}
\label{so-lomon1}
For every subset $K$ of $S$ let
\[x_K=\sum_{w\in X_K}w.\]
Then
\[x_Jx_K=\sum a_{JKL}x_L\]
where $a_{JKL}$ is the number of elements $x\in X_{J}^{-1}\cap X_K$ such that
$x^{-1}W_Jx\cap W_K=W_L$ with $L=x^{-1}Jx\cap K$.
\end{theorem}

The set of all $x_K$ is therefore a basis for an algebra $\Sigma _W$ over
the field of
rationals with integer structure constants $a_{JKL}$.  This algebra
is now known as the {\em descent algebra} of $W$ and much is now
known about its structure
\cite{Atk,G&R,B&B,Ber,BBH&T,GKLL&T}.

Solomon himself began the study of this algebra by determining its
radical, $\rad(\Sigma_{W})$, and some properties of
$\Sigma_{W}/\rad(\Sigma_{W})$.  To describe his results let $\chi_{K}$
be the permutation character of $W$ acting on the right cosets of
$W_{K}$ and let $G_{W}$ be the $Z$-module generated by all
$\chi_{K}$.  Note that each generalised character in $G_{W}$ has
integer values on the elements of $W$.

\begin{theorem} \cite{solomon-mackey}
\label{so-lomon2}
\begin{enumerate}
\item $\rad(\Sigma_{W})$ is spanned by all differences $x_{J}-x_{K}$
where $J$ and $K$ are conjugate subsets of $S$
\item the linear map $\theta$ defined by the images
$\theta(x_{K})=\chi_{K}$ is an algebra homomorphism, and
$\ker\theta=\rad(\Sigma_{W})$
\end{enumerate}
\end{theorem}

Since the structure constants $a_{JKL}$ are integers the $Z$-module
${\cal Z}_W$ spanned by all $x_{K}$ is a subring (an order) of
$\Sigma_{W}$.  This allows us to study the $p$-modular version of the
descent algebra for any prime $p$.  Let $p$ be any prime and let
${\cal P}_W=p{\cal Z}_W$, which is an ideal of ${\cal Z}_W$.
We define $\Sigma (W,p)={\cal Z}_W/{\cal P}_W$, the $p$-modular descent algebra
of  $W$.
$\Sigma (W,p)$ is obviously an algebra over ${\cal F}_{p}$, the field of order
$p$.

Let  $\rho_1$ be the natural projection ${\cal Z}_W\rightarrow
\Sigma(W,p)$
and let $\overline{x}_J=\rho_{1}(x_{J})$.  Then
$$\overline{x}_J\overline{x}_K=\sum\overline{a}_{JKL}\overline{x}_L$$
where $\overline{a}_{JKL}$ is the image of $a_{JKL}$ in ${\cal F}_{p}$.
Furthermore let $\rho_2$ be the map defined on $G_{W}$ which reduces
character values modulo $p$, and let $G(W,p)$ be the image of
$\rho_{2}$.

The map $\phi:\Sigma (W,p)\rightarrow G(W,p)$ defined by
$$ \phi(\rho_1(x))=\rho_2(\theta(x)) \mbox{ for all }x\in{\cal Z}_W$$
is clearly well-defined and is an algebra homomorphism.

 \begin{theorem}
 \label{ma-in2}
 $\rad(\Sigma(W,p))=\ker\phi$. Moreover, $\rad(\Sigma(W,p))$ is spanned by all
 $\overline{x}_J-\overline{x}_K$ where $J,K$ are conjugate subsets
 of $S$, together with all $\overline{x}_J$ for which $p$ divides
 $[N_{W}(W_{J}):J]$.
 \end{theorem}

 This theorem is proved by examining the relationship between the
 structure constants of the algebra and the values of the characters
 in $G_{W}$.


The representation theory of $\Sigma_{W}$ has not been much studied in
general although some results for the Coxeter groups of types $A$ and
$B$ have been found \cite{G&R,B&B,Ber}.  We have shown how
the representation theory of $\Sigma(W,p)$ depends on that
of $\Sigma_{W}$.  The first observation is straightforward: a
representation of $\Sigma_{W}$ over ${\cal F}_{p}$ necessarily has $pZ_{W}$
in its kernel
and so induces a representation of $\Sigma(W,p)$; moreover, every
representation of $\Sigma(W,p)$ arises in this way.  Therefore we may
study the representation theory of $\Sigma(W,p)$ by examining the
$p$-modular representations of $\Sigma_{W}$.  We have done this in the manner
pioneered in group theory: by relating the representations in
characteristic zero to those in characteristic $p$ via a
decomposition matrix.

We can explicitly describe the irreducible representations (they are
all $1$-dimensional and can be read off from the
columns in the parabolic table of marks of the Coxeter group).
Therefore we can give the decomposition matrix
$D$ which stipulates how irreducible representations in
characteristic $0$ behave when reduced modulo $p$.


Of course, the decomposition matrix can be defined in a much more
general context.  Whenever we have a finite dimensional algebra where
reduction modulo $p$ makes sense we can let $\{\tau_{i}\}$ be its irreducible
representations in characteristic zero, $\{\upsilon_{j}\}$ its irreducible
representations in characteristic $p$, and set $d_{ij}$ to be the
multiplicity of $\upsilon_{j}$ as a composition factor of $\tau_{i}$ when
$\tau_{i}$ is reduced
modulo $p$.  In our case the situation is quite simple: as all the
irreducible representations in question are $1$-dimensional these
multiciplicities are either $0$ or $1$.

However, it has proved to be convenient to remain with the more general
situation.
Let $E$ be an algebraically closed complete
local field.  Then $E$ is the field of fractions of a principal ideal
domain $U$, $U$ has a maximal ideal $P$, and $F=U/P$ is a field of
prime characteristic $p$.  Of course, for descent algebras we have
been working over the rational field but, because their irreducible
representations are $1$-dimensional, we can extend to a larger field
without any significant changes.

Let $A$ be an associative algebra over E with an order ${\cal
D}\subset A$.  Then $\bar{\cal D}={\cal D}/P{\cal D}$ is an algebra
over the field $F$.  Moreover, for every ${\cal D}$-module $M$,
$\bar{M}=M/PM$ is a $\bar{\cal D}$-module in a natural way.

Suppose that $P_{1},P_{2},\ldots,P_{s}$ are a full set of principal
indecomposable modules for ${\cal D}$ over $E$ and that
$T_{1},T_{2},\ldots,T_{s}$ are the associated irreducible modules
($P_{i}$ has a unique maximal submodule $\rad(P_{i})$ and
 $T_{i}\cong P_{i}/\rad(P_{i})$).  The Cartan matrix of ${\cal
D}$ is an $s\times s$ matrix $C=(c_{ij})$ whose $(i,j)$ entry is the
number of times that $T_{i}$ occurs as a composition factor of
$P_{i}$.

In an exactly analogous way let $Q_{1},Q_{2},\ldots,Q_{t}$ be a full
set of principal indecomposable modules for $\bar{\cal D}$ over the
field $F$ with associated irreducible modules
$U_{1},U_{2},\ldots,U_{t}$ and let $\tilde{C}$
 be the
Cartan matrix
 of $\bar{\cal D}$.

The algebras ${\cal D}$ and $\bar{\cal D}$ are related by the
decomposition matrix $D$ which describes how each irreducible ${\cal
D}$-module $T_{i}$ behaves when reduced ``mod $p$''.  Specifically,
$D=(d_{ij})$
is an $s\times t$ matrix where $d_{ij}$ is the number of composition
factors of $\bar{T}_{i}$ which are isomorphic to $U_{j}$.

\begin{theorem}
\label{C=DCD}
 $\tilde{C}=D^{T}CD$
\end{theorem}

We have been unable to find a statement of this theorem in the
representation theory literature but it is likely that the result is
not new.  Our proof draws heavily on the approach of
Burrow  \cite{Bur} who considered the case that $A$ was a group algebra
and where $C=I$ since group algebras are semi-simple.
It is plausible that Burrow knew Theorem
\ref{C=DCD} over 30 years ago.  Nevertheless the result deserves to be
better known.


This theorem allows the Cartan matrices of
$\Sigma(W,p)$ to be determined for all the Coxeter groups of type $A$
and $B$ by virtue of results in \cite{G&R,B&B,Ber}.  We
are currently working on the other Coxeter groups.

\begin{thebibliography}{99}

\bibitem{Atk}M.D. Atkinson: Solomon's descent algebra revisited,
 Bull.London Math. Soc. 24 (1992) 545-551.

\bibitem{B&B}F. Bergeron and N. Bergeron: A decomposition of the descent
algebra
of the hyperoctahedral group 1, J. Algebra 148 (1992), 86-97.

\bibitem{Ber}N. Bergeron: A decomposition of the descent algebra
of the hyperoctahedral group 2, J. Algebra 148 (1992), 98-122.

\bibitem{BBH&T}F. Bergeron, N. Bergeron, R.B. Howlett,  and D.E. Taylor,
A decomposition of the descent algebra of a finite
Coxeter group, J. Algebraic Combinatorics 1 (1992), 23-44

\bibitem{Bur}M. Burrow: Representation Theory of Finite Groups,
Academic Press (New York -- London) 1965.

\bibitem{G&R} A.M. Garsia and C.Reutenauer: A decomposition of Solomon's
 descent algebra,Adv. Math. 77 (1989) 189-262.

\bibitem{GKLL&T}I.M. Gelfand, D.Krob, A. Lascoux, B Leclerc, V. Retakh and
 J.-Y. Thibon: Non-commutative symmetric functions, Adv.
 Math. 112 (1995) 218-348.

\bibitem{solomon-mackey}L. Solomon: A Mackey formula in the group ring of a
Coxeter group, J. Algebra 41 (1976), 255-268.

\end{thebibliography}

\end{document}


